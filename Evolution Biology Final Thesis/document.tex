\documentclass[12pt]{article}
\usepackage{ctex}
\usepackage{diagbox}
\usepackage{amsmath}
\usepackage{natbib}
\usepackage{graphicx} 
\usepackage{fontspec}   %修改字
\usepackage{indentfirst} %%首行缩进
\usepackage{xeCJK}
\usepackage{geometry}

\setmainfont{Times New Roman}
\geometry{left=2.2cm,right=2.2cm,top=2.2cm,bottom=2.2cm}
%标题
\title{进化博弈论及模型应用}
%作者
\author{ 马欣悦 \qquad 1653515 \qquad 生命科学与技术学院} 
\date{}

\begin{document}
\maketitle
\begin{Large}
	\par \qquad
\end{Large}
\begin{abstract}
\noindent 博弈论源于经济学家对经济行为的分析,但其精髓在于向着利益最大化做出最佳策略,这一点与达尔文的进化理论相仿,事实上,博弈论不但能很好的运用于进化生物学,解释生物行为,在应用过程中,生物进化也为传统博弈论提供了新思路。本文将介绍进化博弈论的基本概念,并分别介绍这一理论在生物进化及社会科学中的应用模型。
\end{abstract}

\begin{flushleft}
\qquad \quad 关键词: 进化, 博弈论, 进化博弈论, 进化稳态 
\end{flushleft}
\begin{Large}
	\par \qquad
\end{Large}

\begin{center}
	\textbf{Abstract} 
\end{center}
\begin{flushleft}
Game theory was used to explain ecological behavior, the key to which is choosing the best strategy for the greatest benefit. The main idea of game theory is surprisingly compatible with the Darwin's evolutionary theory. Virtually, game theory can explain the biological behavior during evolution as well. Moreover, the application on evolution also provides new-idea for classical game theory. This article will introduce fundamental concept of evolutionary game theory and generate two models based on evolutionary and social issue.
\end{flushleft}

\begin{flushleft}
Keywords: evolution, game theory, evolutionary game theory, evolutionary stable state 
\end{flushleft}


\newpage

\section{前言}
“进化博弈论“是将数学模型应用于生物学的典型例子之一,且由此启发的意义可以广泛延伸到生物进化学之外的学科,如经济学、社会学等,其原因在于[\cite{van2019logics}]:
(1)“进化”这个词本身并非仅局限于生命群体,这个概念本身是可以向外引申的,到社会进化,如信仰、文化等;或者到经济学,如市场竞争、垄断等;
(2)进化博弈论着眼于群体而非个体,策略方法为混合策略(mixed  strategy),这补充了传统博弈论的“纯策略”(pure strategy)中认为“每一个个体都是理性的”的缺陷,因为现实生活中确实很难保证每一个个体的行为都是出于理智的;
(3)进化博弈过程是一个动态变化过程,这继承于生物进化“传代”的特征,上一代的博弈结果(payoff)会作为下一代的博弈策略(strategy)。
\section{发展历史} 
 进化博弈论的概念由R. A. Fisher[\cite{fisher1930genetical}]首次提出,在书中他尝试从博弈论的角度分析动物群体的性别比例,为什么大多数哺乳动物的性别比基本稳定在1:1,在这些群体中,具有交配权的雄性只占全部雄性个体的一小部分,大部分不参与繁衍的雄性个体似乎对群体无甚贡献。Fisher给出的解释是,任意一种性别的个体数量超过另一种性别的个体数时,其本身的适应性(fitness)会下降,而另一种的适应性会上升,这种适应性的不均等最终会推动性别比回到1:1。这个解释虽然没有说明“适应性”指的是什么,而平衡比例又为什么是1:1,但他指出了维稳的过程及原因。
 Fisher的论述反映了进化博弈论的早期思想,但没有系统的将博弈论与进化生物学的概念相结合。1961年R. C. Lewontin在著作“Evolution and the Theory of Games”中第一次提出博弈论在生物进化的应用[\cite{lewontin1961evolution}];1972年,Maynard Smith在“Game Theory and the Evolution of Fighting.”提出了进化稳定策略(evolutionarily stable strategy,简称ESS)的概念;1973年Maynard Smith 和 Price出版“The Logic of Animal Conflict”一书[\cite{smith1973logic}],其中包括经典的“Hawk-Dove”模型。随着该书的出版,ESS的概念开始普及,博弈论开始广泛应用于描述及解释动物行为[\cite{dugatkin2000game}]。1982年Maynard Smith的“Evolution and the Theory of Games”[\cite{smith1982evolution}]及1984年Robert Axelrod的“The Evolution of Cooperation”[\cite{suleiman1996evolution}]出版,进化博弈论这个概念开始受到经济学家及社会学家的关注,并渗透到其他学科当中。

\section{基本概念}
 \subsection{定义}
博弈有三要素:玩家(player),策略(strategy),效益(payoff)。在进化博弈中,生物个体即是玩家,个体的表型即是策略,个体存活率及繁衍能力(fitness)即是效益。博弈过程中,个体的fitness由个体及对手的表型、及环境共同决定。类似传统博弈论,以上的规则可以以二联表的形式展示,种群状态及博弈过程可以用数学语言建立模型,这些概念模型可以辅助我们理解生物行为及进化过程,对选择结果及进化方向做出预测,比如花期、生产期、最适栖息地等[\cite{day2003evolutionary}]。\\
在纯策略博弈中,如果A策略进化稳定,则群体中出现少量B时,一段时间后,B会被逐渐淘汰,群体重新回到A;在混合策略博弈中,如果A策略组合进化稳定,则当任意突变体导致策略组合偏离A,一段时间后,策略组合会重新回到A;我们把这个稳态称为进化稳定状态(evolutionary stable state, ESS)。
假定生物体的策略为z,个体的总效益表示为 $W(z,\overline{z})$。

\begin{displaymath}
W(z_i,\overline{z})=p_1W(z_i,z_1)+p_2W(z_i,z_2)+...+p_nW(z_i,z_n)
\end{displaymath}

其中,$p_1+p_2+...+p_n=1$。在ESS状态下,有
 
\begin{displaymath}
W(z,\overline{z}<W(z^{*},\overline{z})
\end{displaymath}

称 $z^*$ 为进化稳定策略(evolutionary stable state)。达到稳定状态后,倘若出现任意 ${z^*}$,z的分布频率为 $\varepsilon>0$,且满足 $\varepsilon\to{0}$,这时 $\overline z=\varepsilon{z}+(1-\varepsilon)z^*$,满足

\begin{displaymath}
W(z,z^*)\leq{W(z^*,z^*)},\overline{z})
\end{displaymath}

以上便是ESS的数学表示,其中 $2$ 又被称为evolutionary stability,$3$ 被称为convergence stability[\cite{christiansen1991conditions}]。



\subsection{应用及意义}
纳什平衡是博弈论经典且重要的概念之一,当两个玩家的策略处于纳什平衡时,基于对方玩家的策略,没有其他策略可以让玩家得到更多的效益。例如经典的“囚徒困境”:

%%Prisoners' Dilemma
\begin{table}[!htbp]
\centering
\caption{Prisoners' Dilemma payoff matrix}\label{tab:table}%添加标题 设置标签
\begin{tabular}{|c|c|c|}%列格式,居中(c) 居左(l)居右(r)
\hline
{A}{B}&silent&betray\\ %添加斜线表头
\hline % 横线
silent&(-1,-1)&(-3,0)\\%第一行第一列和第二列  中间用&连接
\hline % 横线
betray&(0,-3)&(-2,-2)\\%第二行第一列和第二列  中间用&连接
\hline % 横线
\end{tabular}
\end{table}
任意一方选择了“silent”,另一方的最佳选择(best-response)都会是“betray”,而最终(betray, betray)成为这个博弈的纳什平衡,尽管总体上来看,这似乎并不是一个“明智”的结果。\\
但是对于以下这种情况:
%%Matching Pennies
\begin{table}[!htbp]
\centering
\caption{Matching Pennies payoff matrix}\label{tab:aStrangeTable}%添加标题 设置标签
\begin{tabular}{|c|c|c|}%列格式,居中(c) 居左(l)居右(r)
\hline
\diagbox{A}{B}&Head&Tail\\ %添加斜线表头
\hline % 横线
Head&(0,1)&(1,0)\\%第一行第一列和第二列  中间用&连接
\hline % 横线
Tail&(1,0)&(0,1)\\%第二行第一列和第二列  中间用&连接
\hline % 横线
\end{tabular}
\end{table}
\\
从纯策略的角度来说,四种策略都不是纳什平衡。例如在(Head, Head)时,玩家2知道玩家1的选择是“Head“,为达到最大效益玩家2的策略会从“Head”变成“Tail”;当玩家2变成“Tail”后,玩家1为达到最大效益,策略会从“head”变成“Tail”;当玩家1变成“Tail”时,玩家2的最佳选择又从“Tail”变成了“Head”……如此循环往复。这个情况其实并不少见,比如“剪刀、石头、布”。\\
而通过进化博弈中的混合策略思想,我们就能提出达到纳什平衡的策略。因为混合策略模型着眼于群体或者是多次实验,给出的策略不是(A,B)形式,而是p频率采用Head,1-p频率采用Tail,其中 $0\leq{q}\leq1$。由于玩家策略对称,AB采取的混合策略相同,有

\begin{displaymath}
\left\{
\begin{aligned}
W_A=p[p\times0+(1-p)\times1]+(1-p)\times[p\times1+(1-p)\times0] \\
W_B=p[p\times1+(1-p)\times0]+(1-p)\times[p\times0+(1-p)\times1] \\
\end{aligned}
\right.
\end{displaymath}

化简求得$p_{W_{max}}=0.5$,A,B的最大效益分别为
\begin{displaymath}
\left\{
\begin{aligned}
W_{A(max)}=0.5\\
W_{B(max)}=0.5\\
\end{aligned}
\right.
\end{displaymath}

混合策略除了解决这种纯策略无法分析的博弈情况,还补充了另一种情况,及并非所有玩家都是理智的。这在生物进化过程中液由很好的体现,例如相对劣势的性状面临的并不是在第二代就完全消失,而是逐渐小减少,甚至能够与相对优势性状维持一定比例的稳定分布。“相对”这一概念指出,在现实中影响博弈玩家的策略选择不不单单是游戏规则中明确指出的效益,还有很多其他因素。同时进化博弈的动态模型,例如“replicator dynamics”[\cite{taylor1978evolutionary}],上一代的博弈结果导致下一代的策略分布。这些都为传统博弈论做了补充。

得益于进化博弈论的补充,大量社会行为的分析都采用了这样一种博弈思想,例如利他性[\cite{fletcher2007evolution}],共情能力[\cite{page2002empathy}],社会学习[\cite{kameda2003does}],文化的起源与传播[\cite{enquist2007critical}]。接下来将各介绍一种模型,利用进化博弈论解决生物进化及社会行为问题。


\section{基本模型}
\subsection{Hawk-Dove模型}
Hawk-Dove模型是Maynard Smith和Price所著The Logic of Animal Conflict[\cite{smith1973logic}]的经典例子。下面介绍此类模型的分析方法。
\subsubsection{假设及定义}
\noindent 1.策略(strategy)\\
Hawk:侵略型玩家,直到受伤或者打败对手才停止战斗;\\
Dove:保守型玩家,遇到危险时立刻撤退;\\

\noindent 2.效益(payoff)

%%Hawk and Dove model
\begin{table}[!htbp]
\centering
\caption{Hawk and Dove payoff matrix}\label{tab:aStrangeTable}%添加标题 设置标签
\begin{tabular}{|c|c|c|}%列格式,居中(c) 居左(l)居右(r)
\hline
\diagbox{A}{B}&Hawk&Dove\\ %添加斜线表头
\hline % 横线
Hawk&$\frac{V-C}{2}$&V\\%第一行第一列和第二列  中间用&连接
\hline % 横线
Dove&0&$\frac{V}{2}$\\%第二行第一列和第二列  中间用&连接
\hline % 横线
\end{tabular}
\end{table}

上表中,V为两个玩家相遇时争夺的资源,C为两个玩家势均力敌的斗争造成的资源损失。如上表,一只Dove另一只Dove,二者和平均分V的资源;而当Dove面临Hawk,Hawk的侵略性会让Dove选择逃跑,因此鸽子效益为0而Hawk为V;当Hawk与Hawk针锋相对,战斗最终会使双方都损失 $\frac{1}{2}C$ 的效益。
\subsubsection{纯策略ESS分析} 
$s\in(hawk,dove)$。$w(s_1,s_2)$为$s_1$个体遇到$s_2$时的效益,W(s)为s策略的玩家的总效益,$W_0$为所有玩家初始效益。$\delta$为ESS下的策略,$\mu$为ESS后出现的突变。$\delta$的分布频率为p,且满足$(1-p)\to0$。有如下关系式

\begin{displaymath}
\left\{
\begin{aligned}
W(\delta)=W_0+(1-p)\bigtriangleup W(\delta,\delta)+p\bigtriangleup W(\delta,\mu)\\
W(\mu)=W_0+(1-p)\bigtriangleup F(\mu,\delta)+p\bigtriangleup W(\mu,\mu)\\
\end{aligned}
\right.
\end{displaymath}
当$\delta$为ESS的策略时,满足$W(\delta)>W(\mu)$,由于$(1-p)\to0$,因此只有在
$$\bigtriangleup W(\delta,\delta)>\bigtriangleup W(\mu,\delta)$$
或者
\begin{displaymath}
\left\{
\begin{aligned}
\bigtriangleup W(\delta,\delta)=\bigtriangleup W(\mu,\delta) \\
\bigtriangleup W(\delta,\mu)=\bigtriangleup W(\mu,\mu)\\
\end{aligned}
\right.
\end{displaymath}

\noindent 时成立。\\

有以上结果可知,(Hawk,Dove)的效益关系符合($\delta$,$\mu$)的效益关系,因此从纯策略的角度分析,Hawk策略严格由于Dove。

\subsubsection{混合策略ESS分析} 
群体中(Hawk,Dove)的分布比为(p,1-p)。根据效益表分别计算Hawk和Dove的效益
\begin{displaymath}
\left\{
\begin{aligned}
W_{dove}=(1-p)[p\times0+(1-p)\times\frac{V}{2}] \\
W_{hawk}=p[p\times\frac{V-C}{2}+(1-p)\times V]\\
\end{aligned}
\right.
\end{displaymath}

当且仅当$p=\frac{1}{2}=\frac{V}{V+C}$时,Dove和Hawk的效益均达到最大。


\subsection{Divide-the-cake 模型}
“Divide-the-cake”模型用于解释社会行为中的“公平”,这个模型1996年由Skyrms在Evolution of the Social Contract[\cite{skyrms1996evolution}]中提出。书中描述的游戏规则为:\\

\emph{“现在有一个意外得到的蛋糕,要求是将蛋糕分给所有玩家,已知所有玩家的需求完全相同。如果所有人都同意分蛋糕的方法,每个人都能得到划分到的份额;但只要有人不满划分的方法,没有人能得到蛋糕。”}\\



现在将这个模型简化。由A,B两位玩家分享体积为C的蛋糕。$D_A$,$D_B$为分配方法,$W_A$, $W_B$表示两个玩家分到的蛋糕量。已知 $0\leq W_A,W_B \leq C$。游戏规则为表示如下:

\begin{displaymath}
\left\{
\begin{gathered}
W_A=D_A; W_B=D_B \qquad \hfill {0\leq (W_A+W_B)\leq C}  \\
W_A=W_B=0 \qquad \hfill (W_A+W_B)> C\\
\end{gathered}
\right.
\end{displaymath}\\

凭直觉,我们会觉得$(\frac{C}{2},\frac{C}{2})$就是最佳策略,但是如何解释这种“公平”分配呢?
当玩家是依次做选择时,已知玩家1选择p($0\leq p\leq C$),玩家2的最佳选择是C-p。该博弈的纳什平衡策略有无穷种,为$(p,C-p) (p\in[0,C])$。\\
而更多时候,玩家不清楚自己是那个先做选择的还是后做选择的,这时倘若玩家凭经验推测自己有q的可能先做选择,(1-q)的可能后做选择,玩家的期望效益为:
$$p\times q+(C-p)\times (1-q)\qquad p\in[0,C],q\in[0,1]$$ \\

回到进化博弈的思想。我们假设“Divide-the-cake”实验进行不止一次,且在群体中两两配对进行,下一次的策略取决于上一次的结果(即上面谈到的replicator dynamics)。为方便分析,我们将蛋糕等分为10份,第n代的策略表示为$<p_0,p_1,..p_{10}>_n$,其中$p_i$表示这一代中选择i份蛋糕的频率。一代博弈后,每一种选择的效益表示为$W(p_i)$,值为:
$$W(p_i)=\sum_{j=0}^{10-i}p_j\times i$$
下一代策略则为:
$$p^{'}_{i}=\frac{W(p_i)}{\sum_{j=0}^{10}W(p_j)}$$
因此,只要给出第一代的策略$<p_0,p_1,..p_{10}>_{n=1}$,我们就可以通过计算,并画图看到策略随代数的变化。如下图:
\begin{center}
\includegraphics[height=10cm]{figure.png}
\end{center}

\noindent 其中,$<p_0,p_1,..p_{10}>_{n=1,a}=<0.0544685, 0.236312, 0.0560727, 0.0469244, 0.0562243, 0.0703294,$\\$ 0.151136, 0.162231, 0.0098273, 0.111366, 0.0451093>, $\\$<p_0,p_1,..p_{10}>_{n=1,b}=<0.410376, 0.107375, 0.0253916, 0.116684, 0.0813494, 0.00573677, $\\$0.0277155, 0.0112791, 0.0163166, 0.191699, 0.00607705>$\\

\noindent 有意思的是,并非所有结果都如预期想的$p_5$的频率会最大甚至逼近于0,由进化博弈论思想分析得到的结果可知,最优策略并不唯一,会因为初代的分布不同,使得群体分布最终向不同的方向发展。


\section{总结}
以上是对进化博弈论及模型构建的介绍。通过数学公式解释进化过程,并将数据可视化使得进化过程被清晰展示。进化博弈轮的发展,从引入博弈论的思想辅助进化生物学的研究,到生物进化的规律补充博弈论的不足,两者相得益彰。甚至Maynard Smith在所著的Evolution and the Theory of Games前言中这样写道[\cite{smith1982evolution}]:\\
\par
\emph{
	“[p]aradoxically, it has turned out that game theory is more readily applied to biology than to the field of economic behaviour for which it was originally designed。” 
}

\newpage

\bibliographystyle{apalike}
\bibliography{bibiography}
\end{document}



